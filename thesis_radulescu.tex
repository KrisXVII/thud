%% Le lingue utilizzate, che verranno passate come opzioni al pacchetto babel. Come sempre, l'ultima indicata sar� quella primaria.
%% Se si utilizzano una o pi� lingue diverse da "italian" o "english", leggere le istruzioni in fondo.
\def\thudbabelopt{english,italian}
%% Valori ammessi per target: bach (tesi triennale), mst (tesi magistrale), phd (tesi di dottorato).
%% Valori ammessi per aauheader: '' (vuoto -> nessun header Alpen Adria Univeristat), aics (Department of Artificial Intelligence and Cybersecurity), informatics (Department of Informatics Systems). Il nome del dipartimento � allineato con la versione inglese del logo UniUD.
%% Valori ammessi per style: '' (vuoto -> stile moderno), old (stile tradizionale).
\documentclass[target=bach,aauheader=,style=]{thud}

%% --- Informazioni sulla tesi ---
\course{Informatica}
\title{Implementazione di un sistema di abbonamenti per una piattaforma e-commerce}
\author{Radulescu Cristian}
\supervisor{Prof.\ Vincenzo Riccio}

%% --- Pacchetti consigliati ---
%% pdfx: per generare il PDF/A per l'archiviazione. Necessario solo per la versione finale
\usepackage[a-1b]{pdfx}
%% hyperref: Regola le impostazioni della creazione del PDF... pi� tante altre cose. Ricordarsi di usare l'opzione pdfa.
\usepackage[pdfa]{hyperref}
%% tocbibind: Inserisce nell'indice anche la lista delle figure, la bibliografia, ecc.

%% --- Stili di pagina disponibili (comando \pagestyle) ---
%% sfbig (predefinito): Apertura delle parti e dei capitoli col numero grande; titoli delle parti e dei capitoli e intestazioni di pagina in sans serif.
%% big: Come "sfbig", solo serif.
%% plain: Apertura delle parti e dei capitoli tradizionali di LaTeX; intestazioni di pagina come "big".

\begin{document}
\maketitle

%% Dedica (opzionale)
% \begin{dedication}
% 	Al mio cane,\par per avermi ascoltato mentre ripassavo le lezioni.
% \end{dedication}

%% Ringraziamenti (opzionali)
% \acknowledgements
% Sed vel lorem a arcu faucibus aliquet eu semper tortor. Aliquam dolor lacus, semper vitae ligula sed, blandit iaculis leo. Nam pharetra lobortis leo nec auctor. Pellentesque habitant morbi tristique senectus et netus et malesuada fames ac turpis egestas. Fusce ac risus pulvinar, congue eros non, interdum metus. Mauris tincidunt neque et aliquam imperdiet. Aenean ac tellus id nibh pellentesque pulvinar ut eu lacus. Proin tempor facilisis tortor, et hendrerit purus commodo laoreet. Quisque sed augue id ligula consectetur adipiscing. Vestibulum libero metus, lacinia ac vestibulum eu, varius non arcu. Nam et gravida velit.

%% Sommario (opzionale)
\abstract
Nunc ac dignissim ipsum, quis pulvinar elit. Mauris congue nec leo ornare lobortis. Nulla hendrerit pretium diam nec lobortis. Nullam aliquam laoreet nisl, sit amet facilisis lectus accumsan ut. Duis et elit hendrerit metus venenatis condimentum. Integer id eros molestie, interdum leo sit amet, aliquet metus. Integer fermentum tristique magna, vel luctus neque rhoncus vel. Ut hendrerit et quam et semper. Mauris egestas, odio sed aliquet luctus, magna orci euismod odio, vitae lacinia tellus tellus non lectus. Aliquam urna neque, porta et mattis aliquam, congue sit amet lorem. In ultrices augue sit amet ante vehicula, vitae rhoncus turpis auctor. Donec porta scelerisque eros, at mollis enim imperdiet ut. 

%% Indice
\tableofcontents

%% Lista delle tabelle (se presenti)
%\listoftables

%% Lista delle figure (se presenti)
%\listoffigures

%% Corpo principale del documento
\mainmatter

%% Parte
%% La suddivisione in parti � opzionale; solitamente sono sufficienti i capitoli.
% \part{Parte}

%% Capitolo
\chapter{Introduzione}

introduzione (da scrivere alla fine con contesto, requisiti principali ossia perché stiamo facendo questo software o
stiamo aggiungendo funzionalità, valutazione, conclusioni). Poi elenco contributi per ogni capitolo:
in capitolo 2..., in capitolo 3...

%% Sezione
\section{Titolo della Sezione}
Donec pulvinar neque non lectus vulputate pellentesque. Quisque rutrum arcu velit, in feugiat sapien posuere vel.

%% Sottosezione
\subsection{Sottosezione}

Lorem ipsum dolor sit amet, consectetur adipiscing elit. Aliquam auctor odio sit amet tempor consequat.

\chapter{Background Aziendale}

In questo capitolo verranno esposte le modalità operative adottate in azienda per lo sviluppo della funzionalità discussa.
\section{La scelta dell'azienda}
Nello sviluppo software è importante adottare un approccio ingegneristico e strutturato al fine di sviluppare software di
qualità riducendo i costi e i tempi di sviluppo. Convenire in primis il modello di processo software su cui incentrare lo
sviluppo diventa cruciale. Le alternative principali si dividono fra modelli plan-driven e modelli Agile.\\
Il modello plan-driven si caratterizza per:
\begin{itemize}
    \item una sequenza rigida e ben definita di fasi;
    \item una specifica fortmente strutturata dei requisiti;
    \item ampia produzione di documentazione del software.
\end{itemize}
Passando alla fase successiva a quella attuale tutto ciò che è stato svolto in quest'ultima viene bloccato una volta confermato e non si può più tornare indietro perchè eventuali variazioni nei requisti
durante lo sviluppo potrebbero essere troppo costosi da sviluppare.\\
Il modello Agile invece punta su flessibilità e adattabilità permettendo di:
\begin{itemize}
    \item dare una risposta immediata ai cambiamenti nei requisiti;
    \item coinvolgere il cliente nell'attività di sviluppo;
    \item documentare l'essenziale in modo da concentrarsi sulla scrittura di codice funzionante.
\end{itemize}
L'azienda adotta una variazione del modello agile per cui nei paragrafi seguenti verrà approfondito tale approccio,
con particolare attenzione alle sue implementazioni concrete quali la gestione dei task, i meeting ricorrenti e i strumenti
di supporto.

\chapter{Background Software e Ambiente Operativo}

background software e dell'ambiente operativo (linguaggi, librerie)

\chapter{Funzionalità Sviluppata}

Sistema o funzionalità sviluppata: casi d'uso, deployment diagram, comportamentale tipo activity

\chapter{Valutazione}

Valutazione con testing in RSpec (https://github.com/simplecov-ruby/simplecov), questionario sul software per valutare
la bontà del prodotto e la soddisfazione degli utenti (va bene anche il team sviluppatori) -> da fare ad es su google forms o
microsoft forms con likert e domanda aperta per raccogliere feedback

\chapter{Conclusioni e Sviluppi Futuri}

conclusioni e sviluppi futuri


%% Fine dei capitoli normali, inizio dei capitoli-appendice (opzionali)
\appendix

%\part{Appendici}

% \chapter{Titolo della prima appendice}
% Sed purus libero, vestibulum ut nibh vitae, mollis ultricies augue. Pellentesque velit libero, tempor sed pulvinar non, fermentum eu leo. Duis posuere eleifend nulla eget sagittis. Nam laoreet accumsan rutrum. Interdum et malesuada fames ac ante ipsum primis in faucibus. Curabitur eget libero quis leo porttitor vehicula eget nec odio. Proin euismod interdum ligula non ultricies. Maecenas sit amet accumsan sapien.

%% Parte conclusiva del documento; tipicamente per riassunto, bibliografia e/o indice analitico.
\backmatter

%% Riassunto (opzionale)
%\summary
%Maecenas tempor elit sed arcu commodo, dapibus sagittis leo egestas. Praesent at ultrices urna. Integer et nibh in augue mollis facilisis sit amet eget magna. Fusce at porttitor sapien. Phasellus imperdiet, felis et molestie vulputate, mauris sapien tincidunt justo, in lacinia velit nisi nec ipsum. Duis elementum pharetra lorem, ut pellentesque nulla congue et. Sed eu venenatis tellus, pharetra cursus felis. Sed et luctus nunc. Aenean commodo, neque a aliquam bibendum, mauris augue fringilla justo, et scelerisque odio mi sit amet diam. Nulla at placerat nibh, nec rutrum urna. Donec ut egestas magna. Aliquam erat volutpat. Phasellus vestibulum justo sed purus mattis, vitae lacinia magna viverra. Nulla rutrum diam dui, vel semper mi mattis ac. Vestibulum ante ipsum primis in faucibus orci luctus et ultrices posuere cubilia Curae; Donec id vestibulum lectus, eget tristique est.

%% Bibliografia (praticamente obbligatoria)
\bibliographystyle{plain_\languagename}%% Carica l'omonimo file .bst, dove \languagename � la lingua attiva.
%% Nel caso in cui si usi un file .bib (consigliato)
\bibliography{thud}
\cite{Knu86}
%% Nel caso di bibliografia manuale, usare l'environment thebibliography.

%% Per l'indice analitico, usare il pacchetto makeidx (o analogo).

\end{document}
